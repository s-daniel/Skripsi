\documentclass[a4paper,twoside]{article}
\usepackage[T1]{fontenc}
\usepackage[bahasa]{babel}
\usepackage{graphicx}
\usepackage{graphics}
\usepackage{float}
\usepackage[cm]{fullpage}
\pagestyle{myheadings}
\usepackage{etoolbox}
\usepackage{setspace} 
\usepackage{lipsum} 
\setlength{\headsep}{30pt}
\usepackage[inner=2cm,outer=2.5cm,top=2.5cm,bottom=2cm]{geometry} %margin
% \pagestyle{empty}

\makeatletter
\renewcommand{\@maketitle} {\begin{center} {\LARGE \textbf{ \textsc{\@title}} \par} \bigskip {\large \textbf{\textsc{\@author}} }\end{center} }
\renewcommand{\thispagestyle}[1]{}
\markright{\textbf{\textsc{AIF401/AIF402 \textemdash Rencana Kerja Skripsi \textemdash Sem. Ganjil 2015/2016}}}

\onehalfspacing
 
\begin{document}

\title{\@judultopik}
\author{\nama \textendash \@npm} 

%tulis nama dan NPM anda di sini:
\newcommand{\nama}{Steven Daniel}
\newcommand{\@npm}{2012730021}
\newcommand{\@judultopik}{Penentuan Tingkat Kemacetan Berdasarkan Twitter Menggunakan Jaringan Syaraf Tiruan} % Judul/topik anda
\newcommand{\jumpemb}{1} % Jumlah pembimbing, 1 atau 2
\newcommand{\tanggal}{09/10/2015}
\maketitle

\pagenumbering{arabic}

\section{Deskripsi}
Pertumbuhan jumlah kendaraan di Indonesia setiap tahunnya selalu mengalami peningkatan yang positif, namun pertumbuhan tersebut tidak diimbangi dengan pertumbuhan panjang jalan. Dan hal tersebut mengakibatkan kepadatan lalu-lintas. Ditambah pengendaran yang tidak mematuhi rambu-rambu lalu-lintas akan memperparah kemacetan.\\\\
Twitter adalah media sosial yang memiliki jumlah pengguna di Indonesia mencapai angka 50 juta pada tahun 2015. Pengguna Twitter di Indonesia juga ternyata cukup aktif dalam memberikan \textit{tweets}. Grup analis bernama Semiocast mencatat, pengguna Twitter di kota bandung menyumbang lebih dari 1 miliar \textit{tweets} sepanjang bulan juni 2012. Jumlah \textit{tweets} yang banyak memudahkan kita untuk mencari informasi pada Twitter. Informasi yang dapat kita cari salah satunya mengenai keadaan lalu lintas suatu jalan.\\\\
\textit{API} adalah kepanjangan dari \textit{"Application Programming Interface"}. \textit{API} ialah sebuah cara yang ditetapkan sebuah program untuk menyelesaikan sebuah tugas, biasanya dengan menerima atau memodifikasi data. Twitter menyediakan sebuah \textit{API} yang memberikan hak akses kepada pengembang perangkat lunak untuk membaca dan menulis data dari server Twitter. Dalam pemograman berbahasa Java ada sebuah \textit{library} bernama Twitter4J yang membungkus Twitter API. \textit{Library} ini memudahkan \textit{programmer} Java dalam mengembangkan sebuah perangkat lunak yang memanfaatkan Twitter \textit{API}.\\\\
Jaringan Saraf Tiruan(JST) adalah model komputasi yang terinspirasi dari cara kerja sistem saraf biologi. Sama seperti sistem saraf biologi, JST memiliki neuron-neuron yang dapat meneruskan sinyal apabila sinyal yang dihantarkan melewati nilai tertentu. JST sendiri dapat digunakan untuk menyelesaikan masalah yang rumit. JST juga dapat digunakan untuk membuat sebuah kesimpulan berdasarkan informasi yang ada. Jenis JST yang sering digunakan untuk menyimpulkan suatu informasi ialah Feed Forward Neural Network. Feed Forward Neural Network minimal terdiri dari 3 layer utama yaitu \textit{input layer}, \textit{hidden layer}, dan \textit{output layer}. Feed Forward Neural Network akan menerima sinyal-sinyal dan akan diteruskan ke \textit{output layer}.\\\\
Kemacetan yang sering terjadi berdampak buruk pada kehidupan masyarakat. Solusinya dengan membangun perangkat lunak yang dapat menentukan tingkat kemacetan berdasarkan \textit{tweets} dari Twitter. Perangkat lunak akan mengambil \textit{tweets} menggunakan \textit{library} Java Twitter4J. Perangkat lunak akan memproses \textit{tweets} menggunakan Feed Forward Neural Network. Dengan adanya perangkat lunak ini pengguna dapat menghindari kemacetan.
\section{Rumusan Masalah}
Berdasarkan deskripsi diatas, rumusan masalah adalah sebagai berikut:
\begin{itemize}
	\item Bagaimana melakukan pengambilan \textit{tweets} dari Twitter?
	\item Bagaimana memodelkan teks / kata kunci pada \textit{tweets} di Twitter menjadi sinyal-sinyal pada Jaringan Syaraf Tiruan?
	\item Bagaimana menentukan batas waktu \textit{tweets} yang relevan untuk menjadi input JST?
	\item Bagaimana menentukan jumlah \textit{hidden layer} dan \textit{jumlah neuron} pada JST?
	\item Bagaimana mengimplementasikan JST kedalam bahasa Java?
	\item Apa metode pembelajaran yang akan digunakan pada JST?
	\item Bagaimana cara melatih Jaringan Syaraf Tiruan?
\end{itemize}

\section{Tujuan}
Karya tulis ilmiah ini bertujuan membangun sebuah perangkat lunak yang dapat:
\begin{itemize}
	\item Mengambil \textit{tweets} dari Twitter secara \textit{programmatic}.
	\item Memodelkan teks / kata kunci pada \textit{tweets} menjadi sinyal-sinyal pada JST.
	\item Menentukan batas waktu \textit{tweets} yang relevan untuk menjadi input JST
	\item Mengimplementasikan JST kedalam bahasa Java.
	\item Menentukan jumlah \textit{hidden layer} dan \textit{jumlah neuron} pada JST.
	\item Menentukan metode pembelajaran yang digunakan pada JST.
	\item Melatih Jaringan Syaraf Tiruan.
\end{itemize}


\section{Deskripsi Perangkat Lunak}
Perangkat lunak akhir yang akan dibuat memiliki fitur minimal sebagai berikut:
\begin{itemize}	
	\item Perangkat lunak dapat mengambil tweets dari Twitter.
	\item Perangkat lunak dapat dilakukan proses pembelajaran.
	\item Pengguna dapat mendapatkan informasi tingkat kemacetan suatu jalan.
\end{itemize}

\section{Detail Pengerjaan Skripsi}
Bagian-bagian pekerjaan skripsi ini adalah sebagai berikut :
	\begin{enumerate}
		\item Melakukan studi literatur tentang JST.
		\item Mempelajari \textit{library} Java yang bernama Twitter4j.
		\item Mempelajari karakteristik \textit{tweets} yang di kirimkan pengguna Twitter di Indonesia.
		\item Menentukan input sebagai sinyal JST.
		\item Mengimplementasikan Twitter4j dan menentukan data apa saja yang akan di ambil dari twitter.
		\item Merancang JST kedalam bahasa Java.
	  \item Mengumpulkan input dan output JST sebagai metode pembelajaran
		\item Melakukan uji coba JST dan menyesuaikan konfigurasi JST agar mendapatkan hasil yang paling memuaskan.
		\item Menulis dokumen skripsi
	\end{enumerate}

\section{Rencana Kerja}
rencana saya untuk menyelesaikan skripsi.

\begin{center}
  \begin{tabular}{ | c | c | c | c | l |}
    \hline
    1*  & 2*(\%) & 3*(\%) & 4*(\%) &5*\\ \hline \hline
    1   & 13  & 13  &  &  \\ \hline
    2   & 8 & 8  &   & \\ \hline
    3   & 6  & 6  &  & \\ \hline
    4   & 6  & 4  &  2 & {\footnotesize menyesuaikan input dengan program di S2} \\ \hline
    5   & 15  & 5  & 10 & {\footnotesize sudah menentukan jenis tweet seperti apa di S1} \\ \hline
    6   & 10 &   & 10  & \\ \hline
    7   & 16  &  6 & 10 &   {\footnotesize input mulai dicari dari S1} \\ \hline
    8   & 10  &   &  10  & \\ \hline
    9   & 16  & 8  & 8  & {\footnotesize sebagian bab 1 dan 2, serta bagian awal analisis di S1}\\ \hline
    Total  & 100  & 50  & 50 &  \\ \hline
                          \end{tabular}
\end{center}

Keterangan (*)\\
1 : Bagian pengerjaan Skripsi (nomor disesuaikan dengan detail pengerjaan di bagian 5)\\
2 : Persentase total \\
3 : Persentase yang akan diselesaikan di Skripsi 1 \\
4 : Persentase yang akan diselesaikan di Skripsi 2 \\
5 : Penjelasan singkat apa yang dilakukan di S1 (Skripsi 1) atau S2 (skripsi 2)

\vspace{1cm}
\centering Bandung, \tanggal\\
\vspace{2cm} \nama \\ 
\vspace{1cm}

Menyetujui, \\
\ifdefstring{\jumpemb}{2}{
\vspace{1.5cm}
\begin{centering} Menyetujui,\\ \end{centering} \vspace{0.75cm}
\begin{minipage}[b]{0.45\linewidth}
% \centering Bandung, \makebox[0.5cm]{\hrulefill}/\makebox[0.5cm]{\hrulefill}/2013 \\
\vspace{2cm} Nama: \makebox[3cm]{\hrulefill}\\ Pembimbing Utama
\end{minipage} \hspace{0.5cm}
\begin{minipage}[b]{0.45\linewidth}
% \centering Bandung, \makebox[0.5cm]{\hrulefill}/\makebox[0.5cm]{\hrulefill}/2013\\
\vspace{2cm} Nama: \makebox[3cm]{\hrulefill}\\ Pembimbing Pendamping
\end{minipage}
\vspace{0.5cm}
}{
% \centering Bandung, \makebox[0.5cm]{\hrulefill}/\makebox[0.5cm]{\hrulefill}/2013\\
\vspace{2cm} Nama: \makebox[3cm]{\hrulefill}\\ Pembimbing Tunggal
}
`
\end{document}

