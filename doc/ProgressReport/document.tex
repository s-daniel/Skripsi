\documentclass[a4paper,twoside]{article}
\usepackage[T1]{fontenc}
\usepackage[bahasa]{babel}
\usepackage{graphicx}
\usepackage{graphics}
\usepackage{float}
\usepackage[cm]{fullpage}
\pagestyle{myheadings}
\usepackage{etoolbox}
\usepackage{setspace} 
\usepackage{lipsum} 
\setlength{\headsep}{30pt}
\usepackage[inner=2cm,outer=2.5cm,top=2.5cm,bottom=2cm]{geometry} %margin
% \pagestyle{empty}

\makeatletter
\renewcommand{\@maketitle} {\begin{center} {\LARGE \textbf{ \textsc{\@title}} \par} \bigskip {\large \textbf{\textsc{\@author}} }\end{center} }
\renewcommand{\thispagestyle}[1]{}
\markright{\textbf{\textsc{Laporan Perkembangan Pengerjaan Skripsi\textemdash Sem. Ganjil 2015/2016}}}

\onehalfspacing
 
\begin{document}

\title{\@judultopik}
\author{\nama \textendash \@npm} 

%ISILAH DATA DATA BERIKUT INI:
\newcommand{\nama}{Steven Daniel}
\newcommand{\@npm}{2012730021}
\newcommand{\tanggal}{11/10/2015} %Tanggal pembuatan dokumen
\newcommand{\@judultopik}{Memprediksi Kemacetan di Kota Bandung Menggunakan Jaringan Syaraf Tiruan} % Judul/topik anda
\newcommand{\kodetopik}{PAS3906}
\newcommand{\jumpemb}{1} % Jumlah pembimbing, 1 atau 2
\newcommand{\pembA}{Pascal Alfadian}
\newcommand{\pembB}{-}
\newcommand{\semesterPertama}{7} % semester pertama kali topik diambil, angka 1 dimulai dari sem Ganjil 96/97
\newcommand{\lamaSkripsi}{0.5} % Jumlah semester untuk mengerjakan skripsi s.d. dokumen ini dibuat
\newcommand{\kulPertama}{Skripsi 1} % Kuliah dimana topik ini diambil pertama kali
\newcommand{\tipePR}{B} % tipe progress report :
% A : dokumen pendukung untuk pengambilan ke-2 di Skripsi 1
% B : dokumen untuk reviewer pada presentasi dan review Skripsi 1
% C : dokumen pendukung untuk pengambilan ke-2 di Skripsi 2
\maketitle

\pagenumbering{arabic}

\section{Data Skripsi} %TIDAK PERLU MENGUBAH BAGIAN INI !!!
Pembimbing utama/tunggal: {\bf \pembA}\\
Pembimbing pendamping: {\bf \pembB}\\
Kode Topik : {\bf \kodetopik}\\
Topik ini sudah dikerjakan selama : {\bf \lamaSkripsi} semester\\
Pengambilan pertama kali topik ini pada : Semester {\bf \semesterPertama} \\
Pengambilan pertama kali topik ini di kuliah : {\bf \kulPertama} \\
Tipe Laporan : {\bf \tipePR} -
\ifdefstring{\tipePR}{A}{
			Dokumen pendukung untuk {\BF pengambilan ke-2 di Skripsi 1} }
		{
		\ifdefstring{\tipePR}{B} {
				Dokumen untuk reviewer pada presentasi dan {\bf review Skripsi 1}}
			{	Dokumen pendukung untuk {\bf pengambilan ke-2 di Skripsi 2}}
		}

\section{Detail Perkembangan Pengerjaan Skripsi}
Detail bagian pekerjaan skripsi sesuai dengan rencan kerja/laporan perkembangan terkahir :
	\begin{enumerate}
		\item Melakukan studi literatur mengenai Jaringan Syaraf Tiruan.\\
		{\bf status :} Ada sejak rencana kerja skripsi.\\
		{\bf hasil :} 
		
		\item Mempelajari \textit{library} Java bernama Twitter4J.\\
		{\bf status :} Ada sejak rencana kerja skripsi.\\
		{\bf hasil :} Berhasil melakukan streaming data Twitter dan mengambil tweets dari Twitter dalam 1 minggu terakhir.

		\item Menulis dokumen skripsi\\
		{\bf status :} Ada sejak rencana kerja skripsi.\\
		{\bf hasil :} Penulisan skripsi sampai bab 2

	\end{enumerate}

\section{Pencapaian Rencana Kerja}
Persentase penyelesaian skripsi sampai dengan dokumen ini dibuat dapat dilihat pada tabel berikut :

\begin{center}
  \begin{tabular}{ | c | c | c | c | l | c |}
    \hline
    1*  & 2*(\%) & 3*(\%) & 4*(\%) &5* &6*(\%)\\ \hline \hline
    1   & 13  & 13  &    &  & 13\\ \hline
    2   & 8   & 8   &    &  &  8\\ \hline
    3   & 6   & 6   &    &  & \\ \hline
    4   & 6   & 4   & 2  & {\footnotesize menyesuaikan input dengan program di S2} & \\ \hline
    5   & 15  & 5   & 10 & {\footnotesize sudah menentukan jenis tweet seperti apa di S1} &\\ \hline
    6   & 10  &     & 10 &  & \\ \hline
    7   & 16  &  6  & 10 & {\footnotesize input mulai dicari dari S1} &\\ \hline
    8   & 10  &     & 10 &  &\\ \hline
    9   & 16  & 8   & 8  & {\footnotesize sebagian bab 1 dan 2, serta bagian awal analisis di S1} & 5\\ \hline
    Total  & 100  & 50  & 50 & & 26 \\ \hline
                          \end{tabular}
\end{center}

Keterangan (*)\\
1 : Bagian pengerjaan Skripsi (nomor disesuaikan dengan detail pengerjaan di bagian 5)\\
2 : Persentase total \\
3 : Persentase yang akan diselesaikan di Skripsi 1 \\
4 : Persentase yang akan diselesaikan di Skripsi 2 \\
5 : Penjelasan singkat apa yang dilakukan di S1 (Skripsi 1) atau S2 (skripsi 2)\\
6 : Persentase yang sidah diselesaikan sampai saat ini 

\vspace{1cm}
\centering Bandung, \tanggal\\
\vspace{2cm} \nama \\ 
\vspace{1cm}

Menyetujui, \\
\ifdefstring{\jumpemb}{2}{
\vspace{1.5cm}
\begin{centering} Menyetujui,\\ \end{centering} \vspace{0.75cm}
\begin{minipage}[b]{0.45\linewidth}
% \centering Bandung, \makebox[0.5cm]{\hrulefill}/\makebox[0.5cm]{\hrulefill}/2013 \\
\vspace{2cm} Nama: \pembA \\ Pembimbing Utama
\end{minipage} \hspace{0.5cm}
\begin{minipage}[b]{0.45\linewidth}
% \centering Bandung, \makebox[0.5cm]{\hrulefill}/\makebox[0.5cm]{\hrulefill}/2013\\
\vspace{2cm} Nama: \pemB \\ Pembimbing Pendamping
\end{minipage}
\vspace{0.5cm}
}{
% \centering Bandung, \makebox[0.5cm]{\hrulefill}/\makebox[0.5cm]{\hrulefill}/2013\\
\vspace{2cm} Nama: \pembA \\ Pembimbing Tunggal
}
`
\end{document}